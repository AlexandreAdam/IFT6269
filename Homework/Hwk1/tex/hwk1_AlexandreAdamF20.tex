\documentclass{article}
\usepackage[a4paper, margin=2cm]{geometry}

\usepackage{amsmath}
\usepackage{amstext}
\usepackage{amsthm}
\usepackage{fancyhdr}

\pagestyle{fancy}
\rhead{Alexandre Adam}
\lhead{IFT6269 ― Probabilistic Graphical Models}
\chead{Homework 1}
\rfoot{\today}
\cfoot{\thepage}


\newcommand{\s}{\hspace{0.1cm}}
\numberwithin{equation}{section}
\renewcommand\thesubsection{\alph{subsection})}
\renewcommand\thesubsubsection{\roman{subsubsection}}

\begin{document}

\section{Probability and independence}
\subsection{Decomposition}
We aim to validate
\begin{equation}
        (X \perp Y, W \mid  Z)  \implies (X \perp Y \mid Z)
\end{equation}
\textit{Proof}.  We suppose the statement $(X \perp Y,\ W \mid Z)$ is true. 
It follows from the definition of the conditional independence that
$p(x, y, w \mid z) = p(x\mid z) p(y, w\mid z)$ for all 
$x \in  \Omega_x$, $(y, w) \in \Omega_y \times  \Omega_w$  and $z \in  \Omega_z$.
We then consider the marginalize  $p(x,y,w \mid z)$:
\begin{align*}
        p(x,y|z) &=  \sum_{w\s \in \s \Omega_w}  p(x, y , w \mid z) \\
                 &=  \sum_{w\s \in \s \Omega_w} p(x\mid z)p(y,w \mid z) \\
                 &= p(x \mid z) \sum_{w\s  \in \s  \Omega_w} p(y,w \mid z) \\
                &=  p(x\mid z) p(y\mid  z)
\end{align*}
from which we conclude that  $(X \perp Y \mid Z) \qed$.
By symmetry of the argument,  we can show that ${(X \perp W \mid Z)}$ is true as well.

\subsection{}
We aim to validate 
\begin{equation}\label{eq:1b}
       (X \perp Y \mid Z) \s \text{and} \s (X, Y \perp  W \mid Z) \implies (X \perp W \mid Z)  
\end{equation}
\textit{Proof}. Suppose $(X, Y \perp  W \mid Z)$ and $(X \perp Y \mid Z)$ are true.
We know from the symmetry and decomposition properties of  
the  conditional independence that  
$(X, Y \perp W \mid  Z)  \implies  (W \perp X, Y \mid Z)  \implies  (X \perp  W \mid Z)$.
Therefore $ (X \perp W \mid Z)$ is true $\qed$.


\subsection{}
We aim to validate
\begin{equation}
        (X \perp Y,W \mid Z) \s \text{and} \s (Y \perp W \mid Z) \implies  (X,W \perp Y \mid Z)
\end{equation}
\textit{Proof}. Suppose $(X \perp Y,W \mid Z)$ is true. 
Then it follows from the definition of conditional independence that
\begin{align*}
        p(x,y,w \mid z) = p(x \mid z) p(y,w \mid z) 
\end{align*}
Then assume $(Y \perp W \mid Z)$ is true. The second factor can be factorized
\begin{align*}
        p(x,y,w \mid z) = p(x \mid z) p(y \mid z) p(w \mid z)
\end{align*}
From the decomposition property, we know $(X \perp W \mid Z)$ is true. Thus
\begin{align*}
        p(x,y,w \mid z)  = p(x,w \mid z) p(y \mid z)
\end{align*}
From which we conclude $(X,W \perp Y \mid Z)$ is true $\qed$.

\subsection{}
We aim to validate
\begin{equation}
        (X \perp Y \mid Z) \s \text{and} \s (X \perp Y \mid W) \implies (X \perp Y \mid Z,W)
\end{equation}
\textit{Counter example}. We consider the following R.V.
\begin{enumerate}
        \item X: Person A arrive late for diner;
        \item Y: Person B arrive late for diner;
        \item W: They come from the same city;
        \item Z: They work in the same city.
\end{enumerate}
For this situation, we see that X and Y are conditionally independent when 
given either W or Z. If we know they are from the same city, then they might work 
in different cities and take different route home. Thus knowing person A was late doesn't inform 
us on the probability of person B to arrive late. \par
A similar argument can be made for  $(X \perp Y \mid Z)$. \par
Thus the LHS of the proposition is true, yet the RHS is clearly false in our case. 
Assuming we were given that W, Z are true, this is akin to a geolocalisation of person A and B,
so if we were given that 
person A would be late for diner, then we'd be able to make a good guess that person B would 
be late as well (they both would be impacted by the traffic jam or whatnot).
Thus the proposition is false.

\section{Bayesian inferance and MAP}

\end{document}
